
\documentclass{article}


\usepackage[utf8x]{inputenc}
\usepackage{xcolor}
\usepackage{listings}
\usepackage{amssymb}
\usepackage{amsmath}
\usepackage{stmaryrd}
\usepackage{array}
\usepackage{caption}
\usepackage{bbding}
% \usepackage{fullpage}
\usepackage{comment}
\usepackage{cite}
\usepackage{url}


\pagestyle{plain}

% comments
\newcommand{\MScolor}{blue}
\newcommand\MS[2][r]{\ifx t#1 \textcolor{\MScolor}{[#2]}
  \else \begin{center}\textcolor{\MScolor}{#2} \end{center} \fi}


\begin{document}

This document is written s.t. (most) parts could be directly copied to the paper. Nonetheless some
adaptions most likely will be needed. For instance the text is written as section, whereas it will
be most likely integrated in the Section 5 (Numerical Study). Comments are colored \MS[t]{blue.}


\section{Simulation-Based Optimization}

To provide evidence of the viability of the application of transient queuing approximations in the
setting of order release models, and in particular the SBC method, this section compares results of
the presented SBC approach to simulation optimizations for various scenarios.


We have investigated \MS[t]{3 (TODO)} different demand streams each of which incorporates a certain
demand pattern, e.g.~the first scenario defines a constant demand for periods $7-51$ and no demand
in the other periods, whereas the second scenario provides fluctuations in the demand between
periods $21$ and $81$ and no demand in the other periods, cf. Section \MS[t]{5 (Numerical Study)}.
Each scenario is comprised of $101$ periods of demand data for which a variable length of starting
and ending periods specify a demand of $0$ \MS[t]{true? check other scenarios}. Therefore, we are
able to investigate the behavior of the SBC algorithm in multiple different phases. Besides the
phases defined by the scenarios this also includes when there is no demand, when a demand suddenly
arises and when the demand (unexpectedly) drops to $0$.
%
For each demand stream we compare the results of the SBC optimization as
presented in \MS[t]{Section 4} to the results of simulation-based approaches.

For the later we investigated several techniques to find near-optimal release rates of the demand
stream under investigation. For all approaches the SBC result, possibly adapted to prevent
back-orders in the first evaluation, served as a starting point for finding the simulation-based
optimization solutions. In preliminary experiments we consulted genetic algorithms
\cite{goldberg1988genetic}, interior-point methods~\cite{byrd2000trust,byrd1999interior}, a
mesh-based approach called direct-search or pattern-search~\cite{Hooke:1961:DSS:321062.321069,
  lewis2007implementing,kolda2006stationarity}, and finally the gradient-based SPSA
optimization~\cite{spall1992multivariate,spall1998implementation,kacar2012fitting}. All
simulation-based optimizations used 5000 replications to average the results of the specified
release rates. For the SPSA algorithm we introduced back-order costs of 100 units per order as there
is no possibility to prevent back-orders by constraints.
%
As the best results could be obtained by the SPSA algorithm we selected it as appropriate comparison
measure. The parameter configuration is given in Table~\ref{tbl:params}, see
also~\cite{spall1998implementation}.
%
We adapted the SPSA algorithm in multiple ways. First the degree of freedom is truncated to the last
period in which the demand rate is greater $0$. Then we only allow values in the interval $(0,1.5)$
as possible release rates. And finally we implemented the stopping criteria as presented in the PhD
thesis of Kacar~\cite{kacar2012fitting} which used the algorithm in a similar setting.
%
Despite all the optimizations we observed a high deviation, ranging between a few hours and multiple
days, for the runtime of the SPSA optimization.

% Especially in cases in which the truncating optimization

\begin{table}[t!]
  \centering
  \begin{tabular}{| l | c c c c c c c |}
    \hline
    Parameter & $a$ & $\alpha$ & $A$ & $c$ & $\gamma$ & $\phi$ & $\delta$ \\
    % \hline
    Value & 0.004 & 0.602 & 1000 & 0.1 & 0.101 & 1.0 & 0.05\\
    \hline
  \end{tabular}
  \caption{\label{tbl:params}Parameter setup for the SPSA algorithm as presented
    in~\cite{kacar2012fitting}.}
\end{table}


\pagebreak
\bibliography{references}
\bibliographystyle{plain}
\end{document}


%%% Local Variables:
%%% mode: latex
%%% TeX-master: t
%%% End:
